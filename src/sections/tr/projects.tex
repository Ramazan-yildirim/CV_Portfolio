% ============================================
% PROJECTS - Projeler
% ============================================

\section{Projeler}

\textbf{TEKNOFEST – Otonom Su Altı Aracı (AUV) | Yazılım \& Simülasyon} \hfill \textit{Python, ROS 2, Gazebo, Pixhawk} \\
\vspace{0pt}
\begin{itemize}
    \item TEKNOFEST kapsamında, otonom su altı aracı tasarımı ve simülasyonu üzerine AR-GE çalışmaları yürütüldü.
    \item Araç dinamiği ve kontrol algoritmalarının modellenmesi işlemleri \textbf{ROS 2} ve \textbf{Gazebo} ortamında gerçekleştirildi.
    \item Konum ve durum kestirimi için IMU, basınç ve sonar sensör verilerini içeren \textbf{Sensör Füzyonu} mimarisi üzerine çalışıldı.
    \item Temel hareket ve derinlik kontrol senaryoları, \textbf{Python} tabanlı modüller ile simülasyon ortamında test edildi.
\end{itemize}

\vspace{6pt}

\noindent \textbf{Dijital Diş Dünyası Platformu (Aktif Geliştirme Aşamasında) | Full Stack Geliştirici} \hfill \textit{Vue.js, Laravel, MySQL} \\
\vspace{0pt}
\begin{itemize}
    \item Diş hekimleri, laboratuvarlar ve sektörel firmaları tek bir ekosistemde buluşturan modern ve ölçeklenebilir bir platform geliştirilmektedir.
    \item Kullanıcı deneyimi, etkileşimli arayüzler ve sektöre özel ihtiyaçlar merkeze alınarak uçtan uca bir web uygulaması tasarlanmaktadır.
    \item Frontend tarafında \textbf{Vue.js}, backend tarafında \textbf{Laravel} ve veritabanı olarak \textbf{MySQL} kullanılmaktadır.
\end{itemize}

\vspace{6pt}

\noindent \textbf{Otomatikleştirilmiş LaTeX CV Şablonu | Tasarım \& Otomasyon} \hfill \textit{LaTeX, Docker, CI/CD} \\
\vspace{0pt}
\begin{itemize}
    \item Profesyonel ve modern bir görünüm için özelleştirilebilir \textbf{LaTeX} şablonu tasarlandı ve geliştirildi.
    \item Derleme ortamı bağımsızlığı ve tutarlılık sağlamak amacıyla \textbf{Docker} konteynerizasyonu entegre edildi.
    \item Kaynak kodun modüler yapıda (bölümler halinde) düzenlenmesiyle bakım ve güncellenebilirlik artırıldı.
\end{itemize}

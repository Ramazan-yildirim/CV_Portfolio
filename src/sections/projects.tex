\section*{Projeler}

\datedsubsection{Otonom Sualtı Aracı (AUV) Simülasyonu}{2025}
\textit{Teknolojiler: Python, ROS 2, Gazebo, Simülasyon}

Bu çalışma, TEKNOFEST kapsamında yürütülen otonom bir sualtı aracına yönelik proje sürecinin bir parçasıdır. Projede, simülasyon ortamında temel düzeyde çalışmalar yapılması ve otonom sualtı araçlarının genel çalışma mantığının anlaşılması amaçlanmıştır. Çalışma, yarışma sürecinde değerlendirilerek yarı finale kadar ilerlemiştir.

\vspace{0.2cm} 
\noindent \textbf{Odak Noktaları:}
\begin{itemize}
    \item Simülasyon ortamlarında otonom sistemlere giriş
    \item ROS 2 tabanlı yazılım yapısının genel mantığının incelenmesi
    \item Yarışma süreci içerisinde proje geliştirme deneyimi
\end{itemize}

\vspace{0.5cm}

\datedsubsection{Dijital Diş Dünyası}{2024}
\textit{Teknolojiler: Vue.js, Laravel, MySQL}

Dijital Diş Dünyası, diş hekimliği sektörüne yönelik dijital bir platform fikri üzerine başlatılmış bir web uygulaması çalışmasıdır. Proje kapsamında, içeriklerin ve verilerin tek bir sistem üzerinden sunulabileceği temel bir web yapısının incelenmesi amaçlanmıştır.

\vspace{0.2cm}
\noindent \textbf{Odak Noktaları:}
\begin{itemize}
    \item Vue.js ve Laravel ile temel düzeyde full stack geliştirme
    \item Ön yüz ve arka uç ayrımının anlaşılması
    \item Veritabanı destekli web uygulama yapıları
\end{itemize}

\vspace{0.5cm}

\datedsubsection{Otomatik LaTeX CV ve Portfolyo Oluşturucu}{2024}
\textit{Teknolojiler: LaTeX, Docker}

Bu çalışma, LaTeX kullanılarak CV ve portfolyo dokümanlarının daha düzenli ve tekrar kullanılabilir şekilde hazırlanabilmesini amaçlayan basit bir otomasyon denemesidir. Doküman üretim sürecinin daha sistematik hale getirilmesi hedeflenmiştir.

\vspace{0.2cm}
\noindent \textbf{Odak Noktaları:}
\begin{itemize}
    \item LaTeX ile temel doküman yapısının oluşturulması
    \item Docker kullanılarak derleme sürecinin izole edilmesi
    \item Tekrarlanabilir PDF çıktıları elde edilmesi
\end{itemize}
